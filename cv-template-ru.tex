\documentclass[letterpaper,11pt]{article}

\usepackage{latexsym}
\usepackage[empty]{fullpage}
\usepackage{titlesec}
\usepackage{marvosym}
\usepackage[usenames,dvipsnames]{color}
\usepackage{verbatim}
\usepackage{enumitem}
\usepackage[hidelinks]{hyperref}
\usepackage[T2A]{fontenc}
\usepackage[utf8]{inputenc}
\usepackage[russian, english]{babel}
\usepackage{tabularx}
\usepackage{fontawesome5}
\usepackage{multicol}
\usepackage{graphicx}
\setlength{\multicolsep}{-3.0pt}
\setlength{\columnsep}{-1pt}
\input{glyphtounicode}

\RequirePackage{tikz}
\RequirePackage{xcolor}
\RequirePackage{fontawesome}
\usepackage{tikz}
\usetikzlibrary{svg.path}

\definecolor{cvblue}{HTML}{0E5484}
\definecolor{black}{HTML}{130810}
\definecolor{darkcolor}{HTML}{0F4539}
\definecolor{cvgreen}{HTML}{3BD80D}
\definecolor{taggreen}{HTML}{00E278}
\definecolor{SlateGrey}{HTML}{2E2E2E}
\definecolor{LightGrey}{HTML}{666666}
\colorlet{name}{black}
\colorlet{tagline}{darkcolor}
\colorlet{heading}{darkcolor}
\colorlet{headingrule}{cvblue}
\colorlet{accent}{darkcolor}
\colorlet{emphasis}{SlateGrey}
\colorlet{body}{LightGrey}

% Adjust margins
\addtolength{\oddsidemargin}{-0.6in}
\addtolength{\evensidemargin}{-0.5in}
\addtolength{\textwidth}{1.19in}
\addtolength{\topmargin}{-.7in}
\addtolength{\textheight}{1.4in}

\urlstyle{same}

\raggedbottom
\raggedright
\setlength{\tabcolsep}{0in}

% Sections formatting
\titleformat{\section}{
  \vspace{-4pt}\scshape\raggedright\large\bfseries
}{}{0em}{}[\color{black}\titlerule \vspace{-5pt}]

\pdfgentounicode=1

%-------------------------
% Custom commands
\newcommand{\resumeItem}[1]{
  \item\small{
    {#1 \vspace{-2pt}}
  }
}

\newcommand{\classesList}[4]{
    \item\small{
        {#1 #2 #3 #4 \vspace{-2pt}}
  }
}

\newcommand{\resumeSubheading}[4]{
  \vspace{-2pt}\item
    \begin{tabular*}{1.0\textwidth}[t]{l@{\extracolsep{\fill}}r}
      \textbf{\large#1} & \textbf{\small #2} \\
      \textit{\large#3} & \textit{\small #4} \\
      
    \end{tabular*}\vspace{-7pt}
}

\newcommand{\resumeSubSubheading}[2]{
    \item
    \begin{tabular*}{0.97\textwidth}{l@{\extracolsep{\fill}}r}
      \textit{\small#1} & \textit{\small #2} \\
    \end{tabular*}\vspace{-7pt}
}


\newcommand{\resumeProjectHeading}[2]{
    \item
    \begin{tabular*}{1.001\textwidth}{l@{\extracolsep{\fill}}r}
      \small#1 & \textbf{\small #2}\\
    \end{tabular*}\vspace{-7pt}
}

\newcommand{\resumeSubItem}[1]{\resumeItem{#1}\vspace{-4pt}}

\renewcommand\labelitemi{$\vcenter{\hbox{\tiny$\bullet$}}$}
\renewcommand\labelitemii{$\vcenter{\hbox{\tiny$\bullet$}}$}

\newcommand{\resumeSubHeadingListStart}{\begin{itemize}[leftmargin=0.0in, label={}]}
\newcommand{\resumeSubHeadingListEnd}{\end{itemize}}
\newcommand{\resumeItemListStart}{\begin{itemize}}
\newcommand{\resumeItemListEnd}{\end{itemize}\vspace{-5pt}}

\newcommand\sbullet[1][.5]{\mathbin{\vcenter{\hbox{\scalebox{#1}{$\bullet$}}}}}

%-------------------------------------------
%%%%%%  RESUME STARTS HERE  %%%%%%%%%%%%%%%%%%%%%%%%%%%%


\begin{document}

%----------HEADING----------

\begin{center}
    {\Huge \scshape Тетерин Никита}
    \\ \vspace{2pt}
    Дата рождения: 29.12.2001 \\ \vspace{5pt}
    \href{mailto:cv.ne.teterin@gmail.com}{\raisebox{-0.2\height}\faEnvelope\  \underline{cv.ne.teterin@gmail.com}} ~
    \href{https://github.com/sudotouchwoman}{\raisebox{-0.2\height}\faGithub\ \underline{sudotouchwoman}}~
    \href{https://t.me/sudotouchwoman}{\raisebox{-0.2\height}\faTelegram\             \underline{sudotouchwoman}} ~
    \vspace{5pt}
\end{center}

%-----------EDUCATION-----------
\section{ОБРАЗОВАНИЕ}
\vspace{10pt}
  \resumeSubHeadingListStart
    \resumeSubheading
      {МГТУ им. Н.Э. Баумана, направление 09.03.01}{09.2019 -- настоящее время}
      {\begin{tabular}{@{}l@{}@{}}
      Факультет "Робототехника и комплексная автоматизация",\\
      кафедра систем автоматизированного проектирования, бакалавр.
      \end{tabular}}{Москва, Россия}
  \resumeSubHeadingListEnd

\vspace{3pt}

\section{ДОПОЛНИТЕЛЬНОЕ ОБРАЗОВАНИЕ}
\vspace{10pt}
  \resumeSubHeadingListStart
    \resumeSubheading
      {VK Образование}{09.2021 -- 12.2022}
      {Выпускник основной программы, направление ML-разработчик.}{Москва, Россия}
  \resumeSubHeadingListEnd

\vspace{3pt}

\section{ОПЫТ РАБОТЫ}
\vspace{10pt}
  \resumeSubHeadingListStart
    \resumeSubheading
      {Sber Robotics Lab, Python Developer Intern}{11.2022 -- 02.2023}
        {\begin{tabular}{@{}l@{}@{}}
        Разработал веб-UI для конфигурации и мониторинга логов ESP32. \\
        Исследовал задачу сегментации RGB-D изображений.\\
        \end{tabular}}{Москва, Россия}
  \resumeSubHeadingListEnd

\vspace{3pt}

%-----------PROJECTS-----------
\section{ПРОЕКТЫ}
    \resumeSubHeadingListStart
       \resumeProjectHeading
          {\href{https://github.com/made-ml-in-prod-2022/sudotouchwoman}{\textbf{\large{\underline{Production-ready проект ML-сервиса}}} \href{Project Link}{\raisebox{-0.1\height}\faExternalLink }} $|$ \large{\underline{Python, sklearn, hydra, Flask, Docker, Kubernetes}}}{}
          \resumeItemListStart
            \resumeItem{\normalsize{Организовал гибкий пайплайн обучения и инференса для модели классификации, упаковал в микросервис и развернул в кластере k8s (VKCS).}}
            \resumeItem{\normalsize{\textbf{Ключевые слова: batch/online inference, helm, terraform, MLOps, Github Actions}}}
          \resumeItemListEnd 

      \resumeProjectHeading
          {\href{https://gitlab.com/sudotouchwoman/golang-k8s-ci-example}{\textbf{\large{\underline{CI/CD для сервиса на Go}}} \href{Project Link}{\raisebox{-0.1\height}\faExternalLink }} $|$ \large{\underline{Golang, Gitlab CI, Docker, Kubernetes}}}{}
          \resumeItemListStart
            \resumeItem{\normalsize{Настроил пайплайны для автотестов и линтеров под CRUD-приложение, его последующей сборки, контейнеризации и развертывания на удаленном сервере или в кластере k8s (VKCS) в виде helm чарта.}}
            \resumeItem{\normalsize{\textbf{Ключевые слова: golangci-lint, gitlab-runner, multistage build, kaniko, helm}}}
          \resumeItemListEnd

       \resumeProjectHeading
          {\href{https://github.com/sudotouchwoman/grafana-influx-nmon}{\textbf{\large{\underline{Дашборд метрик производительности}}} \href{Project Link}{\raisebox{-0.1\height}\faExternalLink }} $|$ \large{\underline{Python, InfluxDB, Grafana}}}{}
          \resumeItemListStart
            \resumeItem{\normalsize{Реализовал парсинг и сбор метрик производительности АО сервера с последующей визуализацией результатов в реальном времени.}}
            \resumeItem{\normalsize{\textbf{Ключевые слова: grafana provisioning, nmon, RxPY, Flux}}}
          \resumeItemListEnd 

          \resumeProjectHeading
          {\href{https://github.com/sudotouchwoman/math-misc}{\textbf{\large{\underline{Пет-проекты в области анализа данных}}} \href{Project Link}{\raisebox{-0.1\height}\faExternalLink }} $|$ \large{\underline{Python, numpy, pandas, sklearn, PyTorch, Plotly}}}{}
          \resumeItemListStart
            \resumeItem{\normalsize{Сборник небольших проектов, выполненных в рамках трека ML-разработчик в VK Education либо с целью самообразования.}}
            \resumeItem{\normalsize{\textbf{Ключевые слова: классификация, регрессия, линейные модели, ансамбли, линейное программирование, MLP, RNN, временные ряды}}}
          \resumeItemListEnd 

    \resumeSubHeadingListEnd
\vspace{2pt}

%-----------PROGRAMMING SKILLS-----------
\section{НАВЫКИ}
 \begin{itemize}[leftmargin=0.15in, label={}]
    \small{\item{
     \textbf{\normalsize{Языки программирования:}}{ \normalsize{Python, Golang, C/C++, TypeScript, SQL, \LaTeX{}}} \\
     \textbf{\normalsize{Инструменты:}}{ \normalsize{VS Code, Visual Studio, JupyterLab, PgAdmin, MySQL WB}} \\
     \textbf{\normalsize{Технологии:}}{\normalsize{ Linux, GitHub, Git, CMake, Docker, Kubernetes, Terraform, Grafana, MapReduce, Spark}} \\
     \textbf{\normalsize{Фреймворки:}}{\normalsize{ Flask, PyTorch, PyTorch-Lightning, hydra, sklearn, Plotly, Streamlit, PySpark, Luigi, Gorilla}} \\
     \textbf{\normalsize{Иностранные языки:}}{\normalsize{ Английский C1 (свободное общение, чтение технической документации)}}
     \textbf{\normalsize{Дополнительно:}}{ \normalsize{Имею опыт написания unit-тестов (google-test, pytest) и применения CI (Travis, Gitlab-CI, Github Actions)}} \\
    }}
 \end{itemize}

%-----------ADDITIONAL INFO---------------

\end{document}
